% Fully Connected Neural Network
\begin{tikzpicture}[shorten >=1pt,->,draw=black!50, node distance=\layersep, scale=0.5]
    \tikzstyle{every pin edge}=[<-,shorten <=1pt]
    \tikzstyle{neuron}=[circle,fill=black!25,minimum size=10pt,inner sep=0pt]
    \tikzstyle{input neuron}=[neuron, fill=green!50];
    \tikzstyle{output neuron}=[neuron, fill=red!50];
    \tikzstyle{hidden neuron}=[neuron, fill=blue!50];
    \tikzstyle{annot} = [text width=4em, text centered]

    % Draw the input layer nodes
    \foreach \name / \y in {1,...,4}
    % This is the same as writing \foreach \name / \y in {1/1,2/2,3/3,4/4}
        \node[input neuron, draw=black!100, thick, pin=left:{\scriptsize In{[}\#\y{]}}] (I-\name) at (-\layersep,-1.5-1*\y) {};

    % Draw the hidden layer nodes
    \foreach \name / \y in {1,...,8}
        \path[yshift=0.5cm]
            node[hidden neuron,draw=black!100,thick] (H-\name) at (\layersep,-\y cm) {};

    % Draw the output layer node
    \node[output neuron,draw=black!100,thick,pin={[pin edge={->}]right:{\scriptsize Out{[}\#1{]}}}, right of=H-4] (O1) {};
    \node[output neuron,draw=black!100,thick,pin={[pin edge={->}]right:{\scriptsize Out{[}\#2{]}}}, right of=H-5] (O2) {};

    % Connect every node in the input layer with every node in the
    % hidden layer.
    \foreach \source in {1,...,4}
        \foreach \dest in {1,...,8}
            \path (I-\source) edge (H-\dest);

    % Connect every node in the hidden layer with the output layer
    \foreach \source in {1,...,8}
        \path (H-\source) edge (O1);
    \foreach \source in {1,...,8}
        \path (H-\source) edge (O2);

    % Annotate the layers
    \node[annot,above of=H-1, node distance=1.0cm] (hl) {Hidden layer};
    \node[annot,left of=hl] {Input layer};
    \node[annot,right of=hl] {Output layer};
\end{tikzpicture}