\chapter{Tuning of weight clipping parameter}
\label{chap:W_CLIP}





In the final part of the previous Chapter the procedure of weight clipping was introduced and explained. In this one a systematic approch for weight clipping parameter tuning is developed. The procedure is then tested with several cases, where the difference is in the size of the reference sample. It is important to highlight that in this procedure no signal event is given to the network because we are trying to search the optimal $W$ for which the distribution of $t$ converges after training to a $\chi^2$ distribution with 96 dofs.

Beside searching the optimal weight clipping for each case, another aim is to find a trend for $W$ in order to avoid the tuning when data samples of new sizes are given to the net. The cases studied for the test are listed below in Table \ref{tab:W_CLIP_TEST_SIZES}.

\begin{table}[H]
	\centering
	\begin{tabular}{c c c c}
		\toprule
		\toprule
		n° of test	&	Reference size	&	Data size [\textbf{Zmumu-Zprime}]	&	Data size [\textbf{EFT\_YW06}]	\\
		\midrule
		1			&	100k			&	20k Bkg | 0 Sig				&	20k			\\
		2			&	200k			&	20k Bkg | 0 Sig				&	20k			\\
		3			&	300k			&	20k Bkg | 0 Sig				&	20k			\\
		4			&	500k			&	20k Bkg | 0 Sig				&	20k			\\
		5			&	1000k			&	20k Bkg | 0 Sig				&	20k			\\
		\bottomrule
		\bottomrule
	\end{tabular}
	\caption{Size of the datasets for different tests.}
	\label{tab:W_CLIP_TEST_SIZES}
\end{table}





\section{A criterion for optimal weight clipping}
As suggested in the previous discussion, the optimal weight clipping is reached when the sampled distribution of $t_\mathrm{obs}$ resembles the expected distribution of a $\chi^2$ with 96 dofs. Therefore, after setting a value for $W$ and sampling a certain number of $t$ values, a compatibility test has to be applied with the expected ditribution. A good choice for this purpose is putting the sampled $t$s in a histogram and applying a $\chi^2$ test between the latter and the expected distribution, put in a histogram as well. Now the possible cases are the following:
\begin{itemize}
	\item If the observed $\chi^2$ of the test statistic is beyond an equivalent threshold of $3\sigma$, the observed distribution has approximately converged to the reference one.
	\item If the threshold is not met, the weight clipping has to be lowered or increased depending on the trend of the observed $\chi^2$ of test statistic during training.
	\begin{itemize}
		\item[$\triangleright$] If there is a valley and after reaching a minimum the observed $\chi^2$ begins to grow, the weight clipping has to be lowered.
		\item[$\triangleright$] If there isn't a minimum but the observed $\chi^2$ is still decreasing at the end of training, the weight clipping has to be increased.
	\end{itemize}
\end{itemize}

It is obvious that this method could be computationally very expensive. Finding an optimal value for $W$ requires more trials and every trial requires a certain time as the number of $t$ values needed to sample a distribution is $\gtrsim 10^2$. However, a good strategy to save a lot of time can be easily developed.

The starting point of the search should be the case with the smallest size of reference sample. In fact, training the network in this case requires a relative short time (for instance, $\sim 8~\si{h}$), so more trials can be done without an excessive loss of time. Found the optimal $W$ for this case, the next reference sample in increasing order of size should be used for training. The value of optimal $W$ is expected to be greater than the one in the previous case. The explanation is that giving more reference data to the network is equivalent to have more statistic and so the danger of finding divergences in the training is lower. Hence, $W$ can be set to a greater value as it is a regularization parameter. Moreover, if the size of reference is slightly increased, it is reasonable to think that the optimal weight clipping should be slightly greater as well than in the previous case.





\section{Results for optimal weight clipping search}
Only the best results for each case are reported in the Figures below. Four plots are given:
\begin{itemize}
	\item The final sampled distribution of $t$ along with the expected distribution at 96 dofs.
	\item Several quantiles of the observed $t$ distribution. The quantiles chosen are 2.5\%, 25\%, 50\%, 75\%, 97.5\%. The most important thing to note in this plot is the plateaux reached during training.
	\item $\chi^2$ over training related to the test statistic to assert the compatibility of $t$ distribution and the expected one.
	\item $p$-value over training related to the corresponding value of observed $\chi^2$ in the previous plot. It is calculated considering a $\chi^2$ distribution at $N-1$ dofs, where $N$ is the number of bins.
\end{itemize}




\subsection*{Zmumu-Zprime dataset}
\vspace{-5mm}
\begin{figure}[H]
	\centering
	\includegraphics[width=1.0\textwidth]{Python/W_CLIP/ref100000_bkg20000_sig0/data_ref100000_bkg20000_sig0_wclip2-15.pdf}
	\caption{Optimal $W$ search with \textbf{Zmumu-Zprime} dataset for $N_\mathrm{ref}=100\si{k}$, $N_\mathrm{bkg}=20\si{k}$, $W=2.15$.}
	\label{fig:REF100000_BKG20000_SIG0_WCLIP2.15}
\end{figure}
\vspace{-5mm}
\begin{figure}[H]
	\centering
	\includegraphics[width=1.0\textwidth]{Python/W_CLIP/ref200000_bkg20000_sig0/data_ref200000_bkg20000_sig0_wclip2-4.pdf}
	\caption{Optimal $W$ search with \textbf{Zmumu-Zprime} dataset for $N_\mathrm{ref}=200\si{k}$, $N_\mathrm{bkg}=20\si{k}$, $W=2.4$.}
	\label{fig:REF200000_BKG20000_SIG0_WCLIP2.4}
\end{figure}
\vspace{-5mm}
\begin{figure}[H]
	\centering
	\includegraphics[width=1.0\textwidth]{Python/W_CLIP/ref300000_bkg20000_sig0/data_ref300000_bkg20000_sig0_wclip2-45.pdf}
	\caption{Optimal $W$ search with \textbf{Zmumu-Zprime} dataset for $N_\mathrm{ref}=300\si{k}$, $N_\mathrm{bkg}=20\si{k}$, $W=2.45$.}
	\label{fig:REF300000_BKG20000_SIG0_WCLIP2.45}
\end{figure}
\vspace{-5mm}
\begin{figure}[H]
	\centering
	\includegraphics[width=1.0\textwidth]{Python/W_CLIP/ref500000_bkg20000_sig0/data_ref500000_bkg20000_sig0_wclip2-55.pdf}
	\caption{Optimal $W$ search with \textbf{Zmumu-Zprime} dataset for $N_\mathrm{ref}=500\si{k}$, $N_\mathrm{bkg}=20\si{k}$, $W=2.55$.}
	\label{fig:REF500000_BKG20000_SIG0_WCLIP2.55}
\end{figure}
\vspace{-5mm}
\begin{figure}[H]
	\centering
	\includegraphics[width=1.0\textwidth]{Python/W_CLIP/ref1000000_bkg20000_sig0/data_ref1000000_bkg20000_sig0_wclip2-7.pdf}
	\caption{Optimal $W$ search with \textbf{Zmumu-Zprime} dataset for $N_\mathrm{ref}=1000\si{k}$, $N_\mathrm{bkg}=20\si{k}$, $W=2.7$.}
	\label{fig:REF1000000_BKG20000_SIG0_WCLIP2.7}
\end{figure}






\subsection*{EFT\_YW06 dataset}
\vspace{-5mm}
\begin{figure}[H]
	\centering
	\includegraphics[width=1.0\textwidth]{Python/W_CLIP/ref100000_bkg20000_eft0/data_ref100000_bkg20000_eft0_wclip2-25.pdf}
	\caption{Optimal $W$ search with \textbf{EFT\_YW06} dataset for $N_\mathrm{ref}=100\si{k}$, $N_\mathrm{bkg}=20\si{k}$, $W=2.25$.}
	\label{fig:REF100000_BKG20000_EFT0_WCLIP2.25}
\end{figure}
\vspace{-5mm}
\begin{figure}[H]
	\centering
	\includegraphics[width=1.0\textwidth]{Python/W_CLIP/ref200000_bkg20000_eft0/data_ref200000_bkg20000_eft0_wclip2-4.pdf}
	\caption{Optimal $W$ search with \textbf{EFT\_YW06} dataset for $N_\mathrm{ref}=200\si{k}$, $N_\mathrm{bkg}=20\si{k}$, $W=2.4$.}
	\label{fig:REF200000_BKG20000_EFT0_WCLIP2.4}
\end{figure}
\vspace{-5mm}
\begin{figure}[H]
	\centering
	\includegraphics[width=1.0\textwidth]{Python/W_CLIP/ref300000_bkg20000_eft0/data_ref300000_bkg20000_eft0_wclip2-45.pdf}
	\caption{Optimal $W$ search with \textbf{EFT\_YW06} dataset for $N_\mathrm{ref}=300\si{k}$, $N_\mathrm{bkg}=20\si{k}$, $W=2.45$.}
	\label{fig:REF300000_BKG20000_EFT0_WCLIP2.45}
\end{figure}
\vspace{-5mm}
\begin{figure}[H]
	\centering
	\includegraphics[width=1.0\textwidth]{Python/W_CLIP/ref500000_bkg20000_eft0/data_ref500000_bkg20000_eft0_wclip2-55.pdf}
	\caption{Optimal $W$ search with \textbf{EFT\_YW06} dataset for $N_\mathrm{ref}=500\si{k}$, $N_\mathrm{bkg}=20\si{k}$, $W=2.55$.}
	\label{fig:REF500000_BKG20000_EFT0_WCLIP2.55}
\end{figure}
\vspace{-5mm}
\begin{figure}[H]
	\centering
	\includegraphics[width=1.0\textwidth]{Python/W_CLIP/ref1000000_bkg20000_sig0/data_ref1000000_bkg20000_sig0_wclip2-7.pdf}
	\caption{Optimal $W$ search with \textbf{EFT\_YW06} dataset for $N_\mathrm{ref}=1000\si{k}$, $N_\mathrm{bkg}=20\si{k}$, $W=2.7$.}
	\label{fig:REF1000000_BKG20000_EFT0_WCLIP2.7}
\end{figure}





\subsection*{Optimal weight clipping curve}
\vspace{-5mm}
\begin{figure}[H]
	\centering
	\includegraphics[width=1.0\textwidth]{Python/W_CLIP/W_CLIP_CURVE/w_clip_curve.pdf}
	\caption{Optimal weight clipping curves.}
	\label{fig:W_CLIP_CURVE}
\end{figure}

As it is possible to see in Figure \ref{fig:W_CLIP_CURVE}, the parameter $W$ as a tendency to saturation when $N_\mathrm{ref}$ is much greater than $N_\mathrm{bkg}$. This trend is visible for both type of datasets. Moreover, it is interesting to observe that, fixed the ratio between $N_\mathrm{ref}$ and $N_\mathrm{bkg}$, the optimal weight clipping is approximately the same for the two cases analyzed. Exceptions occur only when the reference is not so populated. This result suggests that for a small reference the anomalies that can be found in the data distributions are different in the two cases. However, these exceptions become irrelevant with a sufficiently large reference, therefore when the statistical fluctuations in the distributions are suppressed by the high statistic.

Finally, these curves showing the optimal weight clipping trend can be used to find an approximation of the optimal $W$ when we want to train the network with new values of $N_\mathrm{ref}$ and $N_\mathrm{bkg}$. What we have to do is just evaluating the ratio $N_\mathrm{ref}/N_\mathrm{bkg} = \xi$, solving the system:
\begin{equation*}
	\begin{cases}
		y = W \left( x = \frac{N_\mathrm{ref}}{N_\mathrm{bkg}} \right)	\\
		x = \xi
	\end{cases}
	\Longrightarrow (x^*,y^*)
\end{equation*}
and taking as optimal $W$ estimate the value of $y^*$.